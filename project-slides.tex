\documentclass[12pt]{beamer}
\mode<presentation> {
	\renewcommand{\familydefault}{\rmdefault}
	\usetheme{CambridgeUS}
	\usepackage[latin1]{inputenc}
	\usefonttheme{professionalfonts}
	\usepackage{times}
	\usepackage{tikz}
	\usepackage{amsmath}
	\usepackage{verbatim}
	\usepackage{enumerate}
	\usepackage{setspace}
	\usetikzlibrary{arrows,shapes}
	\usepackage{amsmath}
	\usepackage{eurosym}
	\usepackage{framed}
	\usepackage{extarrows}
}
\usepackage{graphicx} 
\usepackage{booktabs}

%----------------------------------------------------------------------------------------
%	TITLE PAGE
%----------------------------------------------------------------------------------------

\title{Application of RBF Neural Networks}

\author[Gu Yanzhao]{Instructor: Prof. H\"{a}rdle} % Your name
\institute[]{
	\textsl{Presenter: Gu Yanzhao}\\
	\textsl{ID: 27720151153545}
}
\date[July 16$^{th}, 2016$]{} % Date, can be changed to a custom date

\begin{document}
	
\begin{frame}
	\titlepage 
\end{frame}

\begin{frame}
	\frametitle{Outline} 
	\tableofcontents
\end{frame}

\section{Motivation}
\begin{frame}
	\frametitle{Motivation}
	\begin{itemize}
		\item  \textbf{Definition}:\\
		A neural network is a nonlinear system that maps input variables $x_{1},...,x_{p}$ onto output variables $y_{1},...,y_{q},$i.e it is a nonlinear function
		$$\nu:R^{p}\longrightarrow R^{q}$$
		$$(y_{1},...,y_{q})=\nu(x_{1},...,x_{p}).$$
		\item If one use symmetric kernel functions, e.g. normal-CDF, in this case we speak about RBF-networks.
	\end{itemize}
\end{frame}

\begin{frame}
	\begin{itemize}
		\item \textbf{Objectives}:\\
		\begin{itemize}
			\item [\checkmark] Quantify the risk of an asset using nonlinear $AR(p)-ARCH(q)$ model
			\item [\checkmark] Estimate the conditional volatility using RBF neaural network, which could be seen as a good meassurement for risk
			\item [\checkmark] Illustrate the application of Neural Networks
		\end{itemize}
	\end{itemize}
\end{frame}

\section{Methodology}
\begin{frame}
	\frametitle{Methodology}
	\begin{itemize}
		\item QuantNet-open access code-sharing platform
		\item Quantlet-statistics-related document and program code
		\item Nonlinear $AR(p)-ARCH(q)$ model and RBF neural network
	\end{itemize}
	
\end{frame}

\section{Data}
\begin{frame}
	\frametitle{Data}
	\textbf{Data Set}:
	\begin{itemize}
		\item BP/USD
		\item Jerman 10 year bond yields
		\item gold Krugerrand (SF/Oz)
		\item Commerzbank stocks\\
	\end{itemize}
	\textbf{Remark}:The time interval of the 4 series data is from May.8th,2005 to July.8th,2016, counting for 2895 observations. 
\end{frame}

\section{Emprical Results}
\begin{frame}
	\frametitle{Emprical Results}
	\textbf{Figure 1}
	\includegraphics[height=7cm]{"Rplot"}
\end{frame}
\begin{frame}
	\textbf{Figure2}
	\includegraphics[height=7cm]{"Rplot01"}
\end{frame}
\begin{frame}
	\textbf{Figure3}
	\includegraphics[height=7cm]{"Rplot02"}
\end{frame}

\section{Conclusion}
\begin{frame}
	\frametitle{Conclusions}
	\begin{itemize}
		\item The log return of BP/USD volatiled heavily around the year 2009(after the crisis)
		\item The corresponding estimated conditional volatility of BP/USD is large around the year 2009
		\item The estimated conditional volatility of BP/USD has a decline trend
	\end{itemize}
	\textbf{Conclusion}:The conditional volatility, estimated using RBF neural network, is consistent with the real data of BP/USD.
\end{frame}

\section{}
\begin{frame}
	\frametitle{ }
		\Huge{\centerline{Thanks!}}
\end{frame}
	
	%----------------------------------------------------------------------------------------
\end{document} 